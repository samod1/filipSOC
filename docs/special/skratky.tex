\pagestyle{plain}
\addcontentsline{toc}{chapter}{Zoznam skratiek}
\vspace{0pt plus 2cm}
\chapter*{Zoznam skratiek}
\begin{acronym}
\acro{php}{ Skriptovaci jazyk}
\acro{HTML}{HyperText Markup Language}
\acro{CSS}{Cascade StyleSheets}
\acro{DB}{Databáza}
\acro{UML}{Unified Modelling Language}
\acro{BPMN}{Businnes Procces Modelling and Notation}
\acro{UCD}{Use Case diagram}
\acro{ClD}{Class diagram}
\acro{AcD}{Activity diagram}
\acro{DDL}{Data Definition Language}
\acro{DML}{Data Manipulation Language}
\acro{SQL}{Structured Query Language}
\acro{ASC}{Ascending - od najmensieho po najvacsi}
\acro{DESC}{Descending - od najvacsieho po najmensi}
\acro{BPD}{Business Process Diagram}
\acro{BE}{Back End}
\acro{FE}{Front End}
\end{acronym}

%pouzitie v texte 
% \ac{SKRATKA} - long description (SKRATKA) 
% \ac{SKRATKA} - short - because it is second use 
% \acs{SKRATKA} - short  - this will force short form